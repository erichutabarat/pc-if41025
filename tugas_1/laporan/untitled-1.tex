\documentclass[11pt,a4paper]{article}
%%%%%%%%%%%%%%%%%%%%%%%%% Credit %%%%%%%%%%%%%%%%%%%%%%%%

% template ini dibuat oleh martin.manullang@if.itera.ac.id untuk dipergunakan oleh seluruh sivitas akademik itera.

%%%%%%%%%%%%%%%%%%%%%%%%% PACKAGE starts HERE %%%%%%%%%%%%%%%%%%%%%%%%
\usepackage{graphicx}
\usepackage{caption}
\usepackage{microtype}
\captionsetup[table]{name=Tabel}
\captionsetup[figure]{name=Gambar}
\usepackage{tabulary}
\usepackage{minted}
% \usepackage{amsmath}
\usepackage{fancyhdr}
% \usepackage{amssymb}
% \usepackage{amsthm}
\usepackage{placeins}
% \usepackage{amsfonts}
\usepackage{graphicx}
\usepackage[all]{xy}
\usepackage{tikz}
\usepackage{verbatim}
\usepackage[left=2cm,right=2cm,top=3cm,bottom=2.5cm]{geometry}
\usepackage{hyperref}
\hypersetup{
    colorlinks,
    linkcolor={red!50!black},
    citecolor={blue!50!black},
    urlcolor={blue!80!black}
}
\usepackage{caption}
\usepackage{subcaption}
\usepackage{multirow}
\usepackage{psfrag}
\usepackage[T1]{fontenc}
\usepackage[scaled]{beramono}
% Enable inserting code into the document
\usepackage{listings}
\usepackage{xcolor} 
% custom color & style for listing
\definecolor{codegreen}{rgb}{0,0.6,0}
\definecolor{codegray}{rgb}{0.5,0.5,0.5}
\definecolor{codepurple}{rgb}{0.58,0,0.82}
\definecolor{backcolour}{rgb}{0.95,0.95,0.92}
\definecolor{LightGray}{gray}{0.9}
\lstdefinestyle{mystyle}{
	backgroundcolor=\color{backcolour},   
	commentstyle=\color{green},
	keywordstyle=\color{codegreen},
	numberstyle=\tiny\color{codegray},
	stringstyle=\color{codepurple},
	basicstyle=\ttfamily\footnotesize,
	breakatwhitespace=false,         
	breaklines=true,                 
	captionpos=b,                    
	keepspaces=true,                 
	numbers=left,                    
	numbersep=5pt,                  
	showspaces=false,                
	showstringspaces=false,
	showtabs=false,                  
	tabsize=2
}
\lstset{style=mystyle}
\renewcommand{\lstlistingname}{Kode}
%%%%%%%%%%%%%%%%%%%%%%%%% PACKAGE ends HERE %%%%%%%%%%%%%%%%%%%%%%%%


%%%%%%%%%%%%%%%%%%%%%%%%% Data Diri %%%%%%%%%%%%%%%%%%%%%%%%
\newcommand{\student}{\textbf{Isi Nama Di Sini (Dan Nim Di Sini)}}
\newcommand{\course}{\textbf{Nama Mata Kuliah (Kode Mata Kuliah)}}
\newcommand{\assignment}{\textbf{xxx}}

%%%%%%%%%%%%%%%%%%% using theorem style %%%%%%%%%%%%%%%%%%%%
\newtheorem{thm}{Theorem}
\newtheorem{lem}[thm]{Lemma}
\newtheorem{defn}[thm]{Definition}
\newtheorem{exa}[thm]{Example}
\newtheorem{rem}[thm]{Remark}
\newtheorem{coro}[thm]{Corollary}
\newtheorem{quest}{Question}[section]
%%%%%%%%%%%%%%%%%%%%%%%%%%%%%%%%%%%%%%%%
\usepackage{lipsum}%% a garbage package you don't need except to create examples.
\usepackage{fancyhdr}
\pagestyle{fancy}
\lhead{Nama Mahasiswa di Header (Nim Mahasiswa di Header)}
\rhead{ \thepage}
\cfoot{\textbf{Judul Tugas diketik di sini}}
\renewcommand{\headrulewidth}{0.4pt}
\renewcommand{\footrulewidth}{0.4pt}

%%%%%%%%%%%%%%  Shortcut for usual set of numbers  %%%%%%%%%%%

\newcommand{\N}{\mathbb{N}}
\newcommand{\Z}{\mathbb{Z}}
\newcommand{\Q}{\mathbb{Q}}
\newcommand{\R}{\mathbb{R}}
\newcommand{\C}{\mathbb{C}}
\setlength\headheight{14pt}

%%%%%%%%%%%%%%%%%%%%%%%%%%%%%%%%%%%%%%%%%%%%%%%%%%%%%%%555
\begin{document}
\thispagestyle{empty}
\begin{center}
	\includegraphics[scale = 0.15]{Figure/ifitera-header.png}
	\vspace{0.1cm}
\end{center}
\noindent
\rule{17cm}{0.2cm}\\[0.3cm]
Nama: \student \hfill Tugas Ke: \assignment\\[0.1cm]
Mata Kuliah: \course \hfill Tanggal: Tanggalnya\\
\rule{17cm}{0.05cm}
\vspace{0.1cm}



%%%%%%%%%%%%%%%%%%%%%%%%%%%%%%%%%%%%%%%%%%%%% BODY DOCUMENT %%%%%%%%%%%%%%%%%%%%%%%%%%%%%%%%%%%%%%%%%%%%%
\section{Header Pertama}
    Untuk menggunakan header, anda cukup membuat $\backslash${\tt{section}} pada bagian script anda. Penomoran pada header akan otomatis dibuat. Jika anda membutuhkan line break, anda dapat membubuhkan dua garis miring dengan backslash seperti berikut $\backslash\backslash$

\section{Cara Mencantumkan Link}
    Anda dapat memulai belajar menulis di \LaTeX dengan menulis tugas atau laporan yang sesungguhnya. Pada awalnya mungkin terasa sulit. Namun jika anda dengan tekun tetap berlatih, justru menulis di LaTeX akan membuat anda terbiasa dan malah lebih menyenangkan ketimbang menulis di Microsoft Word.\\
    Jika anda ingin mencantumkan link di \LaTeX anda dapat melakukannya dengan perintah $\backslash${\tt{href}}, misalnya pada tautan berikut ini \href{http://aldi_tob_2000.staff.gunadarma.ac.id/Downloads/files/17359/Membuat+dokumen+dengan+latex.pdf}{(Sumber: PDF Tutorial Belajar Latex)}.
    
\subsection{Sub Header atau Sub Chapter}
     Untuk menggunakan subheader, anda cukup membuat $\backslash${\tt{subsection}} atau bahkan  $\backslash${\tt{subsubsection}} untuk sub-sub bab. Lalu bagaimana untuk membuat subheader atau bahkan header namun tanpa penomoran? Cek keterangan di bawah ini.

\subsection*{Header Tanpa Penomoran}
     Untuk menggunakan subheader tanpa penomoran, anda cukup membuat $\backslash${\tt{subsection*}} atau bahkan  $\backslash${\tt{subsubsection*}} maupun $\backslash${\tt{section*}}.
    
\subsection{Bold, Italic, Plaintext}
\begin{itemize}
    \item Anda dapat membuat cetak tebal dengan perintah $\backslash${\tt{textbf}} seperti \textbf{berikut ini}.
    \item Anda dapat membuat cetak miring dengan perintah $\backslash${\tt{textit}} seperti \textit{berikut ini}. \item Untuk menggaris bawahi, tinggal ketikkan perintah $\backslash${\tt{underline}} seperti \underline{berikut ini}.
    \item Untuk membuat list seperti tulisan ini, gunakan perintah $\backslash${\tt{item}} diantara
\end{itemize}

\subsection{Code Snippets}
    Berikut ini adalah contoh dari penggunaan $\backslash${\tt{begin{lstlisting}}} untuk menulis potongan kode. Dalam kasus ini saya menggunakan bahasa Python. Jika anda menggunakan C atau yang lainnya, tinggal sesuaikan di bagian parameter dari $\backslash${\tt{begin{lstlisting}}}. Anda dapat melihatnya pada code snipptes \ref{labelkode}
    
    \begin{lstlisting}[language=Python, caption=Captionnya tulis di sini class,label={labelkode}]
    
    class SynthiaDataset(Dataset):

    CLASSES = [
        "void", "road", "sidewalk", "building", "wall", "fence", "pole", "traffic light", 
        "traffic sign", "vegetation", "terrain", "sky", "person", "rider", "car", "truck", 
        "bus", "train", "motorcycle", "bicycle", "road lines", "other", "road works"
    ]
    
    def __init__(self, path="../SYNTHIA-SF", classes=None, augmentation=None, preprocessing=None, valid=False):
        self.rootdir = Path(path)
        self.data_imgs, self.data_gts = self.prepare_data(valid,path)
        self.valid = valid

    
        if classes == None:
            classes = self.CLASSES 
        self.class_values = [self.CLASSES.index(cls.lower()) for cls in classes]
        
        self.augmentation = augmentation
        self.preprocessing = preprocessing
    \end{lstlisting}

\section{Memuat Multi-Gambar}
Berikut ini adalah contoh cara untuk memuat multi-gambar.

\begin{figure}[h]
	\centering
	\begin{subfigure}[b]{0.4\textwidth}
		\centering
		\def\svgwidth{\columnwidth}
		\includegraphics[width=1\textwidth]{Figure/aug1.png}
		\caption{Augment Result 1}
		\label{fig:aug-1}
	\end{subfigure}
	\qquad %add desired spacing between images, e. g. ~, \quad, \qquad, \hfill etc. 
	%(or a blank line to force the subfigure onto a new line)
	\begin{subfigure}[b]{0.4\textwidth}
		\centering
		\def\svgwidth{\columnwidth}
		\includegraphics[width=1\textwidth]{Figure/aug2.png}
		\caption{Augment Result 2}
		\label{fig:aug-2}
	\end{subfigure}
	\caption{Augmentation Samples}\label{fig:aug}
\end{figure}


\newpage
\section{Memuat Gambar}
Jika anda tidak ingin memuat multi-gambar, alias satu caption hanya berisi satu gambar. Silahkan tiru cara berikut. Apabila anda merasa gambar anda tidak sesuai pada tempatnya, pelajari cara penempatan gambar pada \href{https://www.overleaf.com/learn/latex/Positioning_of_Figures}{tautan berikut ini}.
\begin{figure}[h]
    \centering
    \includegraphics[width=0.8\textwidth]{Figure/ioutrain.png}
    \caption{Ini Captionnya}
    \label{fig:my_label}
\end{figure}

\section{Memuat Tabel}
Bagian ini adalah bagian yang menurut saya cukup sulit. Namun anda dapat menggunakan bantuan dari table designer yang ada di internet, misalnya \href{https://www.tablesgenerator.com}{TablesGenerator} atau \href{https://www.latex-tables.com}{Latex-Tables}. Ada juga plugin untuk microsoft excel bernama \href{https://ctan.org/tex-archive/support/excel2latex?lang=en}{CTAN}, namun saya jarang mempergunakannya. Contoh tabel dapat dilihat pada Tabel \ref{tab-contoh} berikut.

\begin{table}[h]
\caption{Contoh Tabel}
\label{tab-contoh}
\centering
\resizebox{12cm}{!}{%
\begin{tabular}{ccccccccc}
\hline
\multirow{2}{*}{\textbf{Exp}} & \multirow{2}{*}{\textbf{Mask}} & \multicolumn{3}{c}{\textbf{GT}}            & \multicolumn{3}{c}{\textbf{Proposed}}      & \multirow{2}{*}{\textbf{RMSE}} \\ \cline{3-8}
                              &                                & \textbf{Avg} & \textbf{Max} & \textbf{Min} & \textbf{Avg} & \textbf{Max} & \textbf{Min} &                                \\ \hline
\multirow{2}{*}{1}            & Yes                            & 89.60        & 114.84        & 70.31        & 89.14        & 118.45        & 68.24        & 3.66                           \\
                              & No                             & 90.55        & 112.50        & 75.00        & 89.03        & 109.75        & 71.35        & 3.60                            \\ \hline
\multirow{2}{*}{2}            & Yes                            & 109.84        & 125.98       & 98.44       & 108.62       & 121.30       & 98.26      & 4.04                           \\ 
                              & No                             & 106.62       & 123.44        & 96.09       & 106.48       & 122.19       & 93.37        & 3.95                           \\ \hline
3                             & Yes                            & 74.42        & 94.92       & 62.99        & 73.49       & 102.32        & 60.43       & 3.27                         \\ \hline
Mean                          &                                & 90.61         & 114.34        & 80.57        & 89.70       & 114.80       & 78.33        & 3.63                           \\ \hline
\end{tabular}
}
\end{table}

\section{Referensi dan Daftar Pustaka}
Ini bagian yang sedikit \textit{tricky}. Anda harus memasukkan daftar pustaka anda ke sebuah file berekstensi .bib di bagian kiri dari overleaf ini. Konten dot bib ini dapat anda export dengan mudah, entah itu dari google scholar, mendeley, ataupun manajemen sitasi lainnya.\\\\
Cara penggunaanya pun cukup mudah. Misalnya saat ini saya ingin mensitasi salah satu dokumen yang ada, misalnya wikipedia, saya cukup menuliskan $\backslash${\tt{cite}} yang berisikan cite-key dari entri yang ada di file.bib \cite{Wikipedia_contributors2021-bb}. Contoh lain menuliskan sitasi adalah sebagai berikut \cite{Name2018-hd}.

\newpage
\bibliographystyle{IEEEtran}
\bibliography{Referensi}
\end{document}